\input{../YKY-preamble.tex}

\usepackage[backend=biber]{biblatex}
\bibliography{../AGI-book}

\usepackage[CJKspace]{xeCJK}
\setCJKmainfont[BoldFont=SimHei,ItalicFont=KaiTi]{SimSun}

\title{《香港中立派宣言》}
\author{HK.neutrality@gmail.com}

\setcounter{secnumdepth}{0}		% no section numbers

\begin{document}
\maketitle

{\color{red}[ draft ]}

\section{去除民主选举制}

\begin{itemize}
	\item 民主是错的,民主令社会不断回归到平均值,民主对发展中国家有害,选举 is just a waste of time
	\item {\color{red}[rephrase]} 民主是西方欺压其他民族的手段
	\item 中国在不平等条约下割让香港,所以中国没有义务 honor 中英联合声明,特别是 普选制度。 但在精神上香港中立派 传承了一国两制的安定繁荣 目标。
	\item 西方历史上,民主的发源地 希腊雅典,亦因为民主而战败了。 民主的 Athens 被 Sparta 打败,即著名的 Pelopponesian War.
	\item 柏拉图、亚里士多德 等哲学家 提出 政治的循环 (cycle): 初时社会由暴徒统治 (mob rule),然后强者胜出,变成独裁 (monarchy),独裁的君主越来越暴戾,变成暴君 (tyranny),暴君会被其他贵族驱走,变成 aristocracy,然后贵族日渐腐败变成寡头政治 (oligarchy),寡头又被民众推翻,变成 民主 (democracy),但民主会退化成 暴徒政治 (mob rule),循环重新开始。 
	\item 「中国模式」令中国近10-20年经济起飞,必然做对了某些事。 其实「中国模式」证明了民主对於经济进步是不需要的。 印度有民主,但经济一样起唔到飞。
\end{itemize}

\section{经济自由化 \textbullet 去除家长式管治}

\begin{itemize}
	\item 只有自由市场、经济竞争,才能令 社会进步
	\item 企业联合集资,去除政府管制,并在过渡期获得制度上补偿
	\item 在香港进行 \textbf{土地改革} 不是没有可能的,但土地的经济学和一般财产可能有不同 \cite{Ryan-Collins2017} \cite{Farvacque-Vitkoviac1992} \cite{Blomley2004} \cite{Linklater2013} \cite{Adams2015}
\end{itemize}

\section{全球化 \textbullet 香港中立}

\begin{itemize}
	\item 冷战后香港回归中国,但中国内战造成南北分裂局面,香港变成冷战后政治上的奇异点 (singularity)
	\item 香港中立的优点可以吸引世界上反对种族歧视的人才来港,而不是 吸引一班 racist「鬼佬」来港 坐享特权
\end{itemize}

\printbibliography

\end{document}