\documentclass[16pt]{beamer}

\usepackage[CJKspace]{xeCJK}
\setCJKmainfont[BoldFont=AR PL KaitiM GB,ItalicFont=AR PL KaitiM GB]{SimHei}

%\usepackage{newtxtext,newtxmath}	% use Times Roman font
\usepackage{newtxtext}
\renewcommand{\familydefault}{\sfdefault}
%\usefonttheme{serif}
\usefonttheme{professionalfonts}
%\setbeamertemplate{theorems}[numbered]
\setbeamertemplate{caption}{\insertcaption} 	% no `Figure' prefix before caption

\mode<presentation> {

%\usetheme{default}
%\usetheme{AnnArbor}
%\usetheme{Antibes}
%\usetheme{Bergen}
%\usetheme{Berkeley}
%\usetheme{Berlin}
%\usetheme{Boadilla}
%\usetheme{CambridgeUS}
%\usetheme{Copenhagen}
%\usetheme{Darmstadt}
\usetheme{Dresden}
%\usetheme{Frankfurt}
%\usetheme{Goettingen}
%\usetheme{Hannover}
%\usetheme{Ilmenau}
%\usetheme{JuanLesPins}
%\usetheme{Luebeck}
%\usetheme{Madrid}
%\usetheme{Malmoe}
%\usetheme{Marburg}
%\usetheme{Montpellier}
%\usetheme{PaloAlto}
%\usetheme{Pittsburgh}
%\usetheme{Rochester}
%\usetheme{Singapore}
%\usetheme{Szeged}
%\usetheme{Warsaw}

%\usecolortheme{albatross}
%\usecolortheme{beaver}
%\usecolortheme{beetle}
%\usecolortheme{crane}
%\usecolortheme{dolphin}
%\usecolortheme{dove}
%\usecolortheme{fly}
%\usecolortheme{lily}
%\usecolortheme{orchid}
%\usecolortheme{rose}
%\usecolortheme{seagull}
%\usecolortheme{seahorse}
%\usecolortheme{whale}
\usecolortheme{wolverine}

%\setbeamertemplate{footline} % To remove the footer line in all slides uncomment this line
\setbeamertemplate{footline}[page number] % To replace the footer line in all slides with a simple slide count uncomment this line
\setbeamertemplate{navigation symbols}{} % To remove the navigation symbols from the bottom of all slides uncomment this line
}

\setbeamertemplate{headline}{}
\setbeamersize{text margin left=1mm,text margin right=1mm} 
%\settowidth{\leftmargini}{\usebeamertemplate{itemize item}}
%\addtolength{\leftmargini}{\labelsep}

\usepackage[backend=biber,style=numeric]{biblatex}
\bibliography{../AGI-book}
\bibliography{../Economics}
\renewcommand*{\bibfont}{\footnotesize}
\setbeamertemplate{bibliography item}[text]

\usepackage{graphicx} % Allows including images
\usepackage{tikz-cd}
\usepackage[export]{adjustbox}% http://ctan.org/pkg/adjustbox
\usepackage{verbatim} % comments
% \usepackage{tikz-cd}  % commutative diagrams
% \newcommand{\tikzmark}[1]{\tikz[overlay,remember picture] \node (#1) {};}
% \usepackage{booktabs} % Allows the use of \toprule, \midrule and \bottomrule in tables
% \usepackage{amssymb}  % \leftrightharpoons
\usepackage{wasysym} % frownie face
\usepackage{newtxtext,newtxmath}	% Times New Roman font

\newcommand{\emp}[1]{{\color{blue}#1}}
\newcommand{\vect}[1]{\boldsymbol{#1}}
\newcommand{\tab}{\hspace*{1cm}}
\newcommand*\confoundFace{$\vcenter{\hbox{\includegraphics[scale=0.2]{../confounded-face.jpg}}}$}
\renewcommand{\smiley}{$\vcenter{\hbox{\includegraphics[scale=0.05]{../smiling-face.png}}}$}

%%%%%%%% Make table of contents %%%%%%%

\makeatletter
\renewcommand{\boxed}[1]{\fbox{\m@th$\displaystyle\scalebox{0.9}{#1}$} \,}
\makeatother
\newif\ifframeinlbf
\frameinlbftrue
\makeatletter
\newcommand\listofframes{\@starttoc{lbf}}
\makeatother

\addtobeamertemplate{frametitle}{}{%
	\ifframeinlbf
	\addcontentsline{lbf}{section}{\protect\makebox[2em][l]{%
			\protect\usebeamercolor[fg]{structure}\insertframenumber\hfill}%
		\insertframetitle\par}%
	\else\fi
}

%----------------------------------------------------------------------------------------
%	TITLE PAGE
%----------------------------------------------------------------------------------------

\title[HK neutral party]{\Huge《成立香港中立派》}
\author{HK.neutrality@gmail.com}
%\author{\cc{YKY 甄景贤}{YKY}} % Your name
%\institute[] % Your institution as it will appear on the bottom of every slide, may be shorthand to save space
%{
%Independent researcher, Hong Kong \\ % Your institution for the title page
%\medskip
%\textit{generic.intelligence@gmail.com} % Your email address
%}
\date{\today} % Date, can be changed to a custom date

\begin{document}

\frameinlbffalse
\usebackgroundtemplate{%
	\begin{picture}(0,278)
	\includegraphics[width=\paperwidth]{HK-2.png}
	\end{picture}}
\addtocounter{page}{-1}
\begin{frame}[plain,noframenumbering]
\titlepage
\end{frame}

\usebackgroundtemplate{}
\addtocounter{page}{-1}
\begin{frame}[noframenumbering]
\frametitle{Table of contents}
\listofframes
\vspace*{0.5cm}
多谢 各界支持 \smiley
\end{frame}

%----------------------------------------------------------------------------------------
%	PRESENTATION SLIDES
%----------------------------------------------------------------------------------------

%------------------------------------------------

\frameinlbftrue
\begin{frame}
\frametitle{普选/民主制度 未必适合 香港}

\begin{itemize}
	\item 一人一票的民主制度,令社会不断 回归到平均值,这对於技术发达的西方国家,或许并无大碍,但香港仍处於技术相对落后的后殖民主义阶段,在发展中地区的政治形势更严峻,更需要建立 \emp{meritocracy}

	\item 西方历史上,民主的发源地 希腊雅典,亦没有因为有民主而免於战败。 民主的 Athens 被 Sparta 打败,即著名的 Pelopponesian War
	
	\item 柏拉图、亚里士多德 等哲学家 提出 政治的 \emp{循环} (cycle): \\
	\begin{tikzcd}[ampersand replacement=\&,arrows=Rightarrow,row sep=0em,column sep=small]
		\mbox{mob rule} \arrow{r} \& \mbox{monarchy} \arrow{r} \& \mbox{tyranny} \arrow[d,bend left=90] \\
		\mbox{democracy} \arrow[u,bend left=90] \& \arrow{l} \mbox{oligarchy} \& \arrow{l} \mbox{aristocracy}
	\end{tikzcd}
	% \tab mob rule $\Rightarrow$ monarchy $\Rightarrow$ tyranny $\Rightarrow$ aristocracy $\Rightarrow$ oligarchy $\Rightarrow$ democracy $\Rightarrow$ mob rule
	
	\item 「中国模式」令中国近10-20年经济起飞,必然做对了某些事。 它似乎证明了民主对於经济进步是不必要的。 印度有民主,但经济一样起唔到飞

	\item 中国 在不平等条约下割让香港; 中英联合声明 在历史上发挥了平稳过渡的功能,但它承诺的 普选制度 或已过时
	
	\item Martin Jacques:「回归后香港沿用了殖民政府的系统,但这制度只适合执行殖民统治者的命令,而缺乏一个有效的政治机制」
\end{itemize}
\end{frame}

\begin{frame}
\frametitle{香港人的民族自卑感 \textbullet 全球化 之下的 香港中立}
\begin{itemize}
	\item 冷战后 香港回归中国,但中国 内战 造成南北分裂局面,香港变成冷战后政治上的 奇异点 (\emp{singularity}),这观点与 陈云 的《香港城邦论》类似

	\item 香港中立的优点可以吸引世界上 支持 种族平等 的人材 来港,而不是 吸引一班 racist「鬼佬」来港 坐享特权

	\item 亚洲人 的幼稚化,原因之一是 科技上无法追上欧美,导致强烈的出卖同胞欲望(宁愿做狗不做人)
	
	\item 必需积极地在 本地 和国际上 消除种族歧视,亚洲人才有希望过有尊严的生活
	
	\item 中国 并不需要一个 唯唯诺诺 的香港,正如 老闆 不需要唯唯诺诺的跟班
	
	\item 中央多次明确表示,只要香港不变成颠覆内地政权的基地,香港可以拥有高度自治
	
	\item 有报道指,中央在 2017 年花在广东/香港的 维稳费 已达 1,214亿元 的惊人数字

	% \item 不赞成香港示威者制造矛盾,破坏中港之间的和谐
\end{itemize}
\end{frame}

\begin{frame}
\frametitle{促進 自由竞争 \textbullet 去除 家长式管治}
\begin{itemize}
	\item 只有自由市场、经济竞争,才能令 社会进步
	
	\item 香港要放弃「家长式」管治 (paternalistic governance),政府要「放手」(\emp{deregulation}) 让人们 自己 解决问题
	
	\item 亚洲 流行 家长式 管治,原因是 很多亚洲人像 巨婴。 平均主义者 贪得无厌的诉求,都是基於 不劳而获的心态
	
	\item 美好的生活 包括 有赚钱的自由; 美国梦 (American dream) 指的是 美国人不论出身,都有可能透过努力成为富人
	
	\item 亚洲发展中国家的主要问题,似乎是制度的僵化和官僚过多,或者可以将政府部门转化成私营机构
	
	\item 贪腐 的定义是: ``the abuse of public office for private gain.''  很多官职,表面上为人民服务,实中饱私囊

	\item Deregulation 的做法可以是: 企业联合集资,去除政府管制,并在过渡期获得制度上补偿
\end{itemize}
\end{frame}

\frameinlbftrue
\begin{frame}
\frametitle{土地/房屋 问题}
土地/房屋 问题的相关 actors: (箭咀是钱流动方向)
\begin{equation}
\vcenter{\hbox{\includegraphics[scale=0.6]{housing-solution.png}}}
\nonumber
\end{equation}
可以考虑 Henry George 提出的: 增加 地税 然后派钱给市民的方案; 《香港地产霸权》的作者 Alice Poon 亦支持这一想法
\end{frame}
\frameinlbffalse

\begin{frame}
\frametitle{土地/房屋 问题 (2)}
传统银行 为商业借贷服务的性质 已经变形; 近年银行体系里,大半的资产 是房地产: 
\begin{equation}
\vcenter{\hbox{\includegraphics[scale=1.5]{资产类别价值对比-2.png}}}
\nonumber
\end{equation}
中国人喜欢储蓄,大量的财富储存在房地产; 原则上,不应该剥夺人民的私有财产
\end{frame}

\begin{frame}
\frametitle{土地/房屋 问题 (3)}
\begin{itemize}
	\item 在知识型经济下,地产似乎不再是商业发展的樽颈
	
	\item 土地是有限的 \emp{稀缺资源},地产价值不断攀升的情况下,影响高度密集城市的 可持续发展,这是 徵收地税的理由
	
	\nocite{Ryan-Collins2017}
	\nocite{Farvacque-Vitkoviac1992}
	\nocite{Blomley2004}
	\nocite{Linklater2013}
	\nocite{Adams2015}
	\nocite{简德三2012}
	\nocite{Poon2011}
	\nocite{潘慧娴2010}
	\nocite{OSullivan2012}
	\nocite{Rithmire2015}
	\nocite{Squires2013}
	\nocite{Harvey1996}
	\nocite{Girling1997}
	\nocite{Balia2009}
	\nocite{Peteri2003}
	\nocite{陈云2011}

	\item 在有限的资源下,即使最优的资源分配也无法满足所有人的需要 (wants)

	\item 根据经济学原理,地税 可能会转移给买楼人士; 但似乎可以降低楼价; 不徵税会导致房地产閒置和囤积
	
	\item Henry George: ``The effect of our present system, which taxes a man for values created by his labor and capital, is to put a fine upon industry, and repress improvement.'' 他认为应该徵收 land tax 而不是 property tax

	\item 派钱的理据是: 地税的收入应该属於所有香港人,若由政府分配则有低效率和贪腐的问题
	
	\item 现时,公屋的挑选要求「低收入」不合理,破坏香港人上进的动机,而且抽签 产生「听天由命」心态

	% \item Rent-seeking 不是 land tax 的理由,但稀缺资源是

	% \item 如是者,则 land tax 的目的,必然是将稀缺资源作公平分配

	% \item 问题是 land tax 要收到什么程度?  也许,直到 平均收入 追得上为止。

	% \item 但地税 会 transfer 给租户和住户,导致 楼价 仍然高企。  这样则进一步 增加 basic income.
\end{itemize}
\end{frame}

\frameinlbftrue
\begin{frame}
\frametitle{医疗}
\begin{itemize}
	\item 香港 公立医院 对医护人员的待遇苛刻,「慷他人之慨」,迎合市民贪便宜的心态,实则对公众不利
	
	\item 公立医院 收费过廉,导致 私家医院 被逼提高收费,造成市场上高低两端的「断裂」,和楼市的情况一样
	
	\item 医疗的进步,并不在於廉价劳工,而是需要 技术的进步,包括投入资金 进行 R\&D
	
	\item 政府对医疗的管制其实妨碍了医疗技术的发展
	
	\item 现时在医院死亡的首要原因是医疗失误
\end{itemize}
\end{frame}

\begin{frame}
\frametitle{教育}
\begin{itemize}
	\item 香港 填鸭式教育 已被批评到老掉牙
	
	\item 实际上 香港各大学内,课程、课本、设施等,都和外国大学几乎一模一样
	
	\item 问题是香港(和亚洲很多国家一样)高等教育的普及程度不足
	
	\item 在西方,一般人想读大学 基本上可以找到学位,但香港则「争崩头」 
	
	\item 在「争崩头」情况下,学生/研究生们 专注於争取分数/排名,忘记了求学问的初心 
	
	\item 免费而高质素的网上课程很多,但香港人不重视学问,因为 ``high tech 揩嘢,low tech 捞嘢'' 的环境造成
	
	\item 香港有高科技 R\&D 的公司几乎不存在,没有企业将科技的利润 \emp{回馈} 到学术界,教育的 positive feedback loop 断开了,这情况和外国没有可比性,不能像「南廓吹竽」照抄外国的教育制度
	
	\item 同样地,政府应该「放手」让教育自由化
\end{itemize}
\end{frame}

\frame[allowframebreaks]{
多谢收看 \smiley
% \frametitle{References}
\printbibliography
}

\end{document} 